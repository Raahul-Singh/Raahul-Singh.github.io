% Raahul Singh Resume
% Based on Deedy Resume Template
% IMPORTANT: Compile with XeLaTeX

\documentclass[]{raahul_singh_resume}
\usepackage{fancyhdr}
 
\pagestyle{fancy}
\fancyhf{}
 
\begin{document}

% TITLE NAME
\namesection{}{Raahul Singh}{ \urlstyle{same}
\href{mailto:raahulsingh002@gmail.com}{raahulsingh002@gmail.com} | (+44) 7554539930 | (+91) 8279969625 \\
London | Dehradun
}

% COLUMN ONE
\begin{minipage}[t]{0.33\textwidth} 

% EDUCATION
\section{Education} 

\subsection{IIIT Sri City}
\descript{BTech in Computer Science \\
and Engineering}
\location{ Cum. GPA: 8.43 / 10}
\location{August 2018 - 2022 | India}
\sectionsep

% \subsection{St. Jude's School}
% \location{Grad. May 2017 |  Dehradun, India}
% \sectionsep

% LINKS
\section{Links} 
Website: \href{https://raahulsingh.net}{\bf raahulsingh.net} \\
Github: \href{https://github.com/Raahul-Singh}{\bf Raahul-Singh} \\
GitLab:  \href{https://gitlab.com/rasalghul2}{\bf rasalghul2} \\
LinkedIn:  \href{https://linkedin.com/in/raahulsingh42}{\bf raahulsingh42} \\
% Substack:  \href{https://raahulsingh.substack.com}{\bf pathfinder} \\
% Medium:  \href{https://medium.com/@_hawks_}{\bf @raahulsingh42} \\

% COURSEWORK
\section{Coursework}
\subsection{Undergraduate}
\textbullet{}Advanced Deep Learning and Neural Networks \\
\textbullet{}Artificial Intelligence and Machine Learning \\
\textbullet{}Object-Oriented Programming and Software Design \\
\textbullet{}Information Retrieval and Search Systems \\
\textbullet{}Service-Oriented Architecture and Application Development \\
\sectionsep

% SKILLS
\section{Skills}
\subsection{Programming}
Python \\
\location{Frameworks:}
PyTorch \\
NumPy \\
Pandas \\
SunPy \\
LangChain/Langgraph \\
\sectionsep

% \subsection{Public Speaking}
% \location{English Debate}
% \textbullet{} Won First Prize in International English Debate QUANTA (2016) among speakers from 33 countries at City Montessori School, Lucknow, India \\
% \textbullet{} Accumulated over 7 years of competitive debating experience in Inter-School and District-level English Debates
% \sectionsep

% COLUMN TWO
\end{minipage} 
\hfill
\begin{minipage}[t]{0.66\textwidth} 

% EXPERIENCE
\section{Research and Engineering Experience}

\runsubsection{\lowercase{\href{https://www.phaidra.ai/}}{Phaidra}}
\descript{| Staff AI Research Engineer}
\location{August 2020 – Present}

% Timeline command
\timeline{Timeline: Intern (2020–22) | IC1 (2022–23) | IC2 (2023–24) | Senior (2024–25) | Staff (2025–Present)}

\vspace{\topsep}
\begin{tightemize}
    \item \textbf{Architected the core agentic framework for "Prism," transforming a self-initiated prototype into Phaidra’s flagship Observability system.}
    \item Engineered the agent's tool-execution layer to perform \textbf{automated information fusion}, aggregating heterogeneous industrial data sources for complex multi-step reasoning.
    \item {Designed and implemented the company-wide interactive platform for monitoring production models and agents, providing critical bias and performance insights to Domain Experts.}
    \item {Invented data-agnostic techniques for incorporating domain knowledge into deep neural networks, resulting in more interpretable and physically consistent models.}
    \item {Co-invented a hybrid control architecture \textbf{(Patent US20250021061A1)} that guarantees deterministic safety in mission-critical industrial systems, enabling safe exploration for AI agents.}
    \item {Directed the transition of experimental IP into production by establishing deployment protocols between Research and Engineering teams.}
    \item {Achieved \textbf{15x improvement} in time series prediction accuracy while reducing data requirements by \textbf{30x}, successfully extending prediction horizons to 2x and 3x variables.}
\end{tightemize}
\sectionsep

\runsubsection{\lowercase{\href{https://summerofcode.withgoogle.com/archive/2020/projects/4893913812303872/}}{Google Summer Of Code '20 @ SunPy (OpenAstronomy)}}
\descript{ Student Developer }
\location{May 2020 – July 2020}
\begin{tightemize}
        \item {Developed machine learning models to forecast solar flare probabilities from Active Region data, improving prediction accuracy and reliability.}
        \item {Engineered a Search Events object for seamless querying and matching data across HFC, HEK, and HELIO astronomical databases.}
        \item {{\lowercase{\href{https://gist.github.com/Raahul-Singh/907e5af7c568a94cbb313200f034916f}}{\textbf{\emph{Link to an overview of deliverables.}}}}}
\end{tightemize}
\sectionsep

% \runsubsection{Indian Institute of Technology, Roorkee}
% \descript{\\ Machine Learning Intern @ The Biotech Department }
% \location{May 2019 - July 2019}
% \begin{tightemize}
%         \item{Under the guidance of \lowercase{\href{https://www.iitr.ac.in/~BT/debsrfbt}}{\textbf{Dr. Debabrata Sircar}}, developed and implemented machine learning algorithms to predict fruit shelf-life using biochemical sensor data, achieving significant accuracy improvements.}
%         \item {Conducted comprehensive analysis of fruit volatile chemicals to identify key biological parameters affecting post-harvest storage and nutritional quality.}
% \end{tightemize}
% \sectionsep

% % PROJECTS
% \section{PROJECTS}
% \runsubsection{\lowercase{\href{https://github.com/Raahul-Singh/games-and-ai/tree/master/tictactoe}}{Open Tic Tac Toe and Game Playing AI}}
% \descript{}
% \begin{tightemize}
%         \item {An order of \(10^{37}\) less state searches for a board size of 7 and win score of 3, for the third move of the game vs vanilla minimax. Uses limiting Field of View, spatial locality heuristic and the randomisation of search moves to play near optimally and with a near constant speed irrespective of the board size.}
% \end{tightemize}
% \sectionsep


\section{Publications}
\runsubsection{DETERMINISTIC INDUSTRIAL PROCESS CONTROL}
\descript{Patent US20250021061A1 (Jan 2025)}
\begin{tightemize}
    \item Co-invented a hybrid control architecture that arbitrates between AI agents and local loops to guarantee deterministic constraints in safety-critical systems.
    \item \href{https://patents.google.com/patent/US20250021061A1}{[\textbf{Link to Patent}]}
\end{tightemize}
\sectionsep

\runsubsection{STARKINDLER}
\descript{arXiv Preprint}
\begin{tightemize}
    \item \textbf{Raahul Singh}, Ashutosh Pandey. (2025). \textit{Starkindler: An Uncertainty Aware Objective for Photometric Redshift Estimation.} arXiv preprint arXiv:2512.22566.
    \item Formulated a novel loss function regularised by aleatoric uncertainty, demonstrating significant outlier reduction on SDSS data compared to baseline CNNs.
    \item \href{https://arxiv.org/abs/2512.22566}{[\textbf{Link to Paper}]} \ \ \href{https://github.com/Raahul-Singh/starkindler}{[\textbf{Source Code}]}
\end{tightemize}
\sectionsep



% % Research
% \section{Research}
% \runsubsection{Deterministic industrial process control}
% \descript{\href{https://patents.google.com/patent/US20250021061A1}{US20250021061A1 (Jan 2025)}}
% \begin{tightemize}
%         \item {Co-invented and developed novel control systems for deterministic thermal constraint control, resulting in a published patent application assigned to Phaidra Inc.}
% \end{tightemize}
% \sectionsep

% % PATENTS
% \section{PATENTS}
% \runsubsection{Deterministic industrial process control}
% \descript{\href{https://patents.google.com/patent/US20250021061A1}{US20250021061A1 (Jan 2025)}}
% \begin{tightemize}
%         \item {Co-invented and developed novel control systems for deterministic thermal constraint control, resulting in a published patent application assigned to Phaidra Inc.}
% \end{tightemize}
% \sectionsep

\end{minipage} 
\end{document} 
